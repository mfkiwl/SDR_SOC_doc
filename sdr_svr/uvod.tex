\section{Úvod}
	SDR, nebo-li softwarově definované rádio, je rádiový přijímač (Tato práce se zabývá přijímače, ale SDR může být klidně i vysílač), který ke zpracování přijímaného signálu používá software. Tato metoda má ohromnou výhodu, pokud chceme třeba změnit způsob demodulace přijímaného signálu, tak nemusíme upravovat celé zařízení, ale stačí jen upravit program. Díky tomu může být rádio neustále vylepšováno, aniž bychom museli zasahovat do jeho hardwaru.
	
	\subsection{Koncepce přijímače}
		Tento přijímač byl navržen tak, aby přijímanou stanici směšoval do základního pásma. Díky tomu se dá výstup s SDR dát na linkový vstup libovolné zvukové karty, která signál digitalizuje a umožní jej softwarově zpracovat.
		
	
  		
  		