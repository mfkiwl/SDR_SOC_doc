\section*{Závěr}
\addcontentsline{toc}{section}{Závěr}
\indent\indent Cílem mé práce bylo vytvořit SDR přijímač pro KV pásmo. To se mi nakonec povedlo, i když musím uznat že cesta k cíli nebyla vůbec jednoduchá. Přijímač je určený pro příjem radioamatérského pásma $20~m$. Většinu času jsem strávil studiem níže uvedených materiálů, protože tohle je můj první přijímač. O to větší z něj mám ale radost. Přijímači jsem změřil selektivitu vstupního napětí, aby se ověřil vstupní aktivní filtr popsaný ve druhé kapitole. Výsledky měření jsou shrnuty v obrázku číslo 3. Během vývoje byly největší problémy s návrhem pásmových propustí, vystřídal jsem velké množství železo-prachových a feritových jared, tvarů a zapojení, než jsem došek k filtrům uvedeným v této práci. Další komplikací při vývoji byl směšovač navinutý na toroidech z materiálu Amidon T-44-2. Tyto toroidy měly totiž moc malé $A_L$, díky čemuž na nich namotané cívky měly na $14~MHz$ moc malou reaktanci. Konkrétně $22~\Omega$. Nakonec se osvědčily obyčejné tlumivky z železo-prachových jader, které měly reaktanci na $14~MHz$ několik set ohmů. Když byly tyto problémy překonány, tak bylo třeba navrhnout plošné spoje. Na těchto plošných spojích se zařízení ale chovalo jinak a tak bylo nutné poupravit hodnoty kondenzátorů jinak než byly na prototypech. Na finálních verzích plošných spojů se filtry chovají mnohem ostřeji než na testovacích destičkách. to je způsobeno zkrácením všech cest na minimum, které jednostranné plošné spoje umožní. Zařízení jsem nakonec s tátovou pomocí zapouzdřil do rackové krabice.

Na řídící destičce je nachystaný konektor pro USART, už mám nachystaný i další modul pro převod USART na USB, díky němuž bude v blízké době možné zařízení kompletně ovládat z osobního počítače. To by umožnilo se vhodným software použít toto zařízení jako maják. Tento maják by fungoval následovně. Pomocí internetu by se připojil operátor vzdálené radio stanice k uživatelskému rozhraní pro obsluhu. Pomocí svého vysílače by vyslal například svůj volací znak. Pokud by byla radiostanice v dosahu, tak by mu aplikace přehrála audio záznam příjmu, tak jak jde slyšet v mém přijímači. Díky tomu by si mohl uživatel vzdálené radiostanice udělat obrázek o tom, jaký má jeho vysílač dosah.