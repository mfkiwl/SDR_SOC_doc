\section*{Závěr}
\addcontentsline{toc}{section}{Závěr}
\indent\indent Cílem mé práce bylo navrhnout a vyrobit SDR přijímač pro KV páasmo $20~m$. Jelikož jsem nikdy neřešil VF zařízení, byla pro mne stavba tohoto přijímače obtížná, ale nakonec se mi podařilo rádio dokončit. Většinu času jsem věnoval studiu literatury, návrhu, sestavení a měření bloků pásmových propustí, VF zesilovačů a směšovače. Největší problémy jsem měl s výběrem vhodných materiálů pro výrobu indukčností pro filtry a směšovač. Ve filtrech jsem po vyzkoušení různých materiálů, použil jakostní železoprachová jádra AMIDON T-44-2. Filtry sestavené na těchto jádrech měly výborné parametry, zejména strmost. Díky jejich dobrým parametrům při použití ve filtru, jsem je chtěl použít i pro směšovač, nicméně pro tento účel jsou tato jádra nevhodná. Po několika pokusech s jádry AMIDON jsem použil širokopásmová feritová jádra a směšovač začal pracovat. Problém byl zejména v tom, že železoprachová jádra AMIDON měla nízkou hodnotu $A_L$, a díky tomu byla reaktance cívek přibližně $22~\Omega$. Při použití jader s vyšší hodnotou $A_L$ se zvýšila reaktance až na několik set ohmů. Veškerý vývoj a počáteční měření, jsem prováděl na měřících přípravcích sestavených na DPS metodou MANHATAN. Po osazení součástek do finálního návrhu DPS, bylo pak ještě nutné upravit hodnoty některých kapacit. Toto bylo způsobeno zejména zkrácením spojů realizovaných na finální DPS a úpravou vývodů indukčností. 

Na řídící desce je připraven konektor s vyvedeným rozhraním USART, kterým je možné 
připojit osobní počítač a ovládat celé rádio dálkově. Připojení je možné pomocí USART rozhraní, nebo přes USB rozraní s využitím převodníku USB/USART.  Takto připojené rádio je pak s pomocí připojení do sítě internet a vhodného softwatre možné využít třeba jako maják. Tento maják by sloužil k orientačnímu měření dosahu vysílače, kdy obsluha vysílače přes internet naladí přijímaný kmitočet, zvolí druh modulace a nastaví maják na příjem. Poté vyšle signál, například volací znak, a následně si může přehrát záznam a zkontrolovat kvalitu příjmu.
